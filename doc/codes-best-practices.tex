
\documentclass[conference,10pt,compsocconf,onecolumn]{IEEEtran}
%\documentclass{acm_proc_article-sp}
%\documentclass{sig-alternate}
% Add the compsoc option for Computer Society conferences.
%
% If IEEEtran.cls has not been installed into the LaTeX system files,
% manually specify the path to it like:
% \documentclass[conference]{../sty/IEEEtran}

% *** CITATION PACKAGES ***
%
\usepackage{cite}
% cite.sty was written by Donald Arseneau
% V1.6 and later of IEEEtran pre-defines the format of the cite.sty package
% \cite{} output to follow that of IEEE. Loading the cite package will
% result in citation numbers being automatically sorted and properly
% "compressed/ranged". e.g., [1], [9], [2], [7], [5], [6] without using
% cite.sty will become [1], [2], [5]--[7], [9] using cite.sty. cite.sty's
% \cite will automatically add leading space, if needed. Use cite.sty's
% noadjust option (cite.sty V3.8 and later) if you want to turn this off.
% cite.sty is already installed on most LaTeX systems. Be sure and use
% version 4.0 (2003-05-27) and later if using hyperref.sty. cite.sty does
% not currently provide for hyperlinked citations.
% The latest version can be obtained at:
% http://www.ctan.org/tex-archive/macros/latex/contrib/cite/
% The documentation is contained in the cite.sty file itself.
%
\usepackage{setspace}
\usepackage{wrapfig}
\usepackage{color}
\usepackage{listing}
\usepackage{listings}

\lstset{ %
frame=single,
language=C,
captionpos=b,
columns=fullflexible,
morekeywords={aesop,pwait,pbranch,pbreak},
numbers=left,
basicstyle=\scriptsize\ttfamily,
breaklines=true,
framexleftmargin=0em,
boxpos=c,
resetmargins=true,
xleftmargin=6ex
%basicstyle=\footnotesize
}


\usepackage[pdftex]{graphicx}
%\usepackage{graphicx}
% declare the path(s) where your graphic files are
\graphicspath{{./}{data/beagle/}}
% and their extensions so you won't have to specify these with
% every instance of \includegraphics
\DeclareGraphicsExtensions{.pdf,.jpeg,.png}

% *** GRAPHICS RELATED PACKAGES ***
%
\ifCLASSINFOpdf
 % \usepackage[pdftex]{graphicx}
 % declare the path(s) where your graphic files are
 % \graphicspath{{../pdf/}{../jpeg/}}
 % and their extensions so you won't have to specify these with
 % every instance of \includegraphics
 % \DeclareGraphicsExtensions{.pdf,.jpeg,.png}
\else
 % or other class option (dvipsone, dvipdf, if not using dvips). graphicx
 % will default to the driver specified in the system graphics.cfg if no
 % driver is specified.
 % \usepackage[dvips]{graphicx}
 % declare the path(s) where your graphic files are
 % \graphicspath{{../eps/}}
 % and their extensions so you won't have to specify these with
 % every instance of \includegraphics
 % \DeclareGraphicsExtensions{.eps}
\fi

% *** ALIGNMENT PACKAGES ***
%
%\usepackage{array}
% Frank Mittelbach's and David Carlisle's array.sty patches and improves
% the standard LaTeX2e array and tabular environments to provide better
% appearance and additional user controls. As the default LaTeX2e table
% generation code is lacking to the point of almost being broken with
% respect to the quality of the end results, all users are strongly
% advised to use an enhanced (at the very least that provided by array.sty)
% set of table tools. array.sty is already installed on most systems. The
% latest version and documentation can be obtained at:
% http://www.ctan.org/tex-archive/macros/latex/required/tools/

%\usepackage{eqparbox}
% Also of notable interest is Scott Pakin's eqparbox package for creating
% (automatically sized) equal width boxes - aka "natural width parboxes".
% Available at:
% http://www.ctan.org/tex-archive/macros/latex/contrib/eqparbox/

\usepackage{algorithm}
\usepackage[noend]{algorithmic}

% *** SUBFIGURE PACKAGES ***
\usepackage[tight,footnotesize]{subfigure}
% \usepackage{subfigure}
% subfigure.sty was written by Steven Douglas Cochran. This package makes it
% easy to put subfigures in your figures. e.g., "Figure 1a and 1b". For IEEE
% work, it is a good idea to load it with the tight package option to reduce
% the amount of white space around the subfigures. subfigure.sty is already
% installed on most LaTeX systems. The latest version and documentation can
% be obtained at:
% http://www.ctan.org/tex-archive/obsolete/macros/latex/contrib/subfigure/
% subfigure.sty has been superceeded by subfig.sty.

%\usepackage[caption=false]{caption}
%\usepackage[font=footnotesize]{subfig}
% subfig.sty, also written by Steven Douglas Cochran, is the modern
% replacement for subfigure.sty. However, subfig.sty requires and
% automatically loads Axel Sommerfeldt's caption.sty which will override
% IEEEtran.cls handling of captions and this will result in nonIEEE style
% figure/table captions. To prevent this problem, be sure and preload
% caption.sty with its "caption=false" package option. This is will preserve
% IEEEtran.cls handing of captions. Version 1.3 (2005/06/28) and later 
% (recommended due to many improvements over 1.2) of subfig.sty supports
% the caption=false option directly:
%\usepackage[caption=false,font=footnotesize]{subfig}
%
% The latest version and documentation can be obtained at:
% http://www.ctan.org/tex-archive/macros/latex/contrib/subfig/
% The latest version and documentation of caption.sty can be obtained at:
% http://www.ctan.org/tex-archive/macros/latex/contrib/caption/

% *** PDF, URL AND HYPERLINK PACKAGES ***
%
\usepackage{url}
% url.sty was written by Donald Arseneau. It provides better support for
% handling and breaking URLs. url.sty is already installed on most LaTeX
% systems. The latest version can be obtained at:
% http://www.ctan.org/tex-archive/macros/latex/contrib/misc/
% Read the url.sty source comments for usage information. Basically,
% \url{my_url_here}.

% *** Do not adjust lengths that control margins, column widths, etc. ***
% *** Do not use packages that alter fonts (such as pslatex).         ***
% There should be no need to do such things with IEEEtran.cls V1.6 and later.
% (Unless specifically asked to do so by the journal or conference you plan
% to submit to, of course. )

% correct bad hyphenation here
\hyphenation{op-tical net-works semi-conduc-tor}

\begin{document}
\title{CODES Best Practices}

%\author{\IEEEauthorblockN{Someone\IEEEauthorrefmark{1}} \\
%\IEEEauthorblockA{\IEEEauthorrefmark{1}Somewhere}
%}

%\numberofauthors{6} %  in this sample file, there are a *total*


% use for special paper notices
%\IEEEspecialpapernotice{(Invited Paper)}

% use arabic rather than roman numerals for table references
\renewcommand{\thetable}{\arabic{table}}

% make the title area
\maketitle

\begin{abstract}
This document outlines best practices for developing models in the
CODES/ROSS framework.  The reader should already be familiar with ROSS
and discrete event simulation in general; those topics are covered in the primary
ROSS documentation.
%
The main purpose of this document is to help the reader produce
CODES models in a consistent, modular style so that componets can be more
easily shared and reused.  It also includes a few tips to help avoid common
simulation bugs.
\end{abstract}

\section{CODES: modularizing models}

This section covers some of the basic principles of how to organize model
components to be more modular and easier to reuse across CODES models.

\subsection{Units of time}

ROSS does not dictate the units to be used in simulation timestamps.
The \texttt{tw\_stime} type could represent any time unit
(e.g. days, hours, seconds, nanoseconds, etc.).  When building CODES
models you should \emph{always treat timestamps as double precision floating
point numbers representing nanoseconds}, however.
All components within a model must agree on the time units in order to
advance simulation time consistently.  Several common utilities in the
CODES project expect to operate in terms of nanoseconds.

\subsection{Organizing models by LP types}

ROSS allows you to use as many different LP types as you would like to
construct your models.  Try to take advantage of this as much as possible by
organizing your simulation so that each component of the system that you are
modeling is implemented within its own LP type.  For example, a storage
system model might use different LPs for hard disks, clients, network
adapters, and servers.  There are multiple reasons for dividing up models
like this:

\begin{itemize}
\item General modularity: makes it easier to pull out particular components
(for example, a disk model) for use in other models.
\item Simplicitity: if each LP type is only handling a limited set of
events, then the event structure, state structure, and event handler
functions will all be much smaller and easier to understand.
\item Reverse computation: it makes it easier to implement reverse
computation, not only because the code is simpler, but also because you can
implement and test reverse computation per component rather than having to
apply it to an entire model all at once before testing.
\end{itemize}

It is also important to note that you can divide up models not just by
hardware components, but also by functionality, just as
you would modularize the implementation of a distributed file system.  For
example, a storage daemon might include separate LPs for replication, failure
detection, and reconstruction.  Each of those LPs can share the same network
card and disk resources for accurate modeling of resource usage.  They key
reason for splitting them up is to simplify the model and to encourage
reuse.

One hypothetical downside to splitting up models into multiple LP types is that it likely
means that your model will generate more events than a monolithic model
would have.  Remember that \emph{ROSS is really efficient at generating and
processing events}, though!  It is usually a premature optimization to try to optimize a model by
replacing events with function calls in cases where you know the necessary
data is available on the local MPI process.  Also recall that any information
exchanged via event automatically benefits by shifting burden for
tracking/retaining event data and event ordering into ROSS rather than your
model.  This can help simplify reverse computation by breaking complex
operations into smaller, easier to understand (and reverse) event units with
deterministic ordering.

TODO: reference example, for now see how the LPs are organized in Triton
model.

\subsection{Protecting data structures}

ROSS operates by exchanging events between LPs.  If an LP is sending
an event to another LP of the same type, then in general it can do so
by allocating an event structure (e.g. \texttt{tw\_event\_new()}),
populating the event structure, and transmitting it
(e.g. \texttt{tw\_event\_send()}).  If an LP is sending an event to
another LP of a \emph{different} type, however, then it should use an
explicit API to do so without exposing the other LP's event structure
definition.  Event structures are not a robust API for exchanging data
across different LP types.  If one LP type accesses the event (or state)
structure of another LP type, then it entangles the two components such
that one LP is dependent upon the internal architecture of another LP.
This not only makes it difficult to reuse components, but also makes it
difficult to check for incompatibilities at compile time.  The compiler
has no way to know which fields in a struct must be set before sending
an event.

For these reasons we encourage that a) each LP be implemented in a separate
source file and b) all event structs and state structs
be defined only within those source files.  They should not be exposed in external
headers.  If the definitions are placed in a header then it makes it
possible for those event and state structs to be used as an ad-hoc interface
between LPs of different types.

Section~\ref{sec:completion} will describe alternative mechanisms for
exchanging information between different LP types.

TODO: reference example, for now see how structs are defined in Triton
model.

\subsection{Techniques for exchanging information and completion events
across LP types}
\label{sec:completion}

TODO: fill this in.

Send events into an LP using a C function API that calls event\_new under
the covers.

Indicate completion back to the calling LP by either delivering an opaque 
message back to the calling LP (that was passed in by the caller in a void*
argument), or by providing an API function for 2nd LP type to
use to call back (show examples of both).

\section{CODES: common utilities}

TODO: point out what repo each of these utilities can be found in.

\subsection{codes\_mapping}
\label{sec:mapping}

TODO: pull in Misbah's codes-mapping documentation.

\subsection{modelnet}

TODO: fill this in.  Modelnet is a network abstraction layer for use in
CODES models.  It provides a consistent API that can be used to send
messages between nodes using a variety of different network transport
models.  Note that modelnet requires the use of the codes-mapping API,
described in previous section.

\subsection{lp-io}

TODO: fill this in.  lp-io is a simple API for storing modest-sized
simulation results (not continous traces).  It handles reverse computation
and avoids doing any disk I/O until the simulation is complete.  All data is
written with collective I/O into a unified output directory.  lp-io is
mostly useful for cases in which you would like each LP instance to report
statistics, but for scalability and data management reasons those results
should be aggregated into a single file rather than producing a separate
file per LP.

\subsection{codes-workload generator}

TODO: fill this in.  codes-workload is an abstraction layer for feeding I/O
workloads into a simulation.  It supports multiple back-ends for generating
those I/O events; data could come from a trace file, from Darshan, or from a
synthetic description.

This component is under active development right now, not complete yet.  The
synthetic generator is probably pretty solid for use already though.

\subsection{codes\_event\_new}

TODO: fill this in.  codes\_event\_new is a small wrapper to tw\_event\_new
that checks the incoming timestamp and makes sure that you don't exceed the
global end timestamp for ROSS.  The assumption is that CODES models will
normally run to a completion condition rather than until simulation time
runs out, see later section for more information on this approach.

\section{CODES: reproducability and model safety}

TODO: fill this in.  These are things that aren't required for modularity,
but just help you create models that produce consistent results and avoid
some common bugs.

\subsection{Event magic numbers}

TODO: fill this in.  Put magic numbers at the top of each event struct and
check them in event handler.  This makes sure that you don't accidentally
send the wrong event type to an LP.

\subsection{Small timestamps for LP transitions}

TODO: fill this in.  Sometimes you need to exchange events between LPs
without really consuming significant time (for example, to transfer
information from a server to its locally attached network card).  It is
tempting to use a timestamp of 0, but this causes timestamp ties in ROSS
which might have a variety of unintended consequences.  Use
codes\_local\_latency for timing of local event transitions to add some
random noise, can be thought of as bus overhead or context switch overhead.

\section{ROSS: general tips}

\subsection{Organizing event structures}

TODO: fill this in.  The main idea is to use unions to organize fields
within event structures.  Keeps the size down and makes it a little clearer
what variables are used by which event types.

\subsection{Avoiding event timestamp ties}

TODO: fill this in.   Why ties are bad (hurts reproducability, if not
accuracy, which in turn makes correctness testing more difficult).  Things
you can do to avoid ties, like skewing initial events by a random number
generator.

\subsection{Validating across simulation modes}

TODO: fill this in.  The general idea is that during development you should
do test runs with serial, parallel conservative, and parallel optimistic
runs to make sure that you get consistent results.  These modes stress
different aspects of the model.

\subsection{Reverse computation}

TODO: fill this in.  General philosophy of when the best time to add reverse
computation is (probably not in your initial rough draft prototype, but it
is best to go ahead and add it before the model is fully complete or else it
becomes too daunting/invasive).

Other things to talk about (maybe these are different subsections):
\begin{itemize}
\item propagate and maintain as much state as possible in event structures
rather than state structures
\item rely on ordering enforced by ROSS (each
reverse handler only needs to reverse as single event, in order)
\item keeping functions small 
\item building internal APIs for managing functions with reverse functions
\item how to handle queues
\end{itemize}

\subsection{How to complete a simulation}

TODO: fill this in.  Most core ROSS examples are design to intentionally hit
the end timestamp for the simulation (i.e. they are modeling a continuous,
steady state system).  This isn't necessarily true when modeling a
distributed storage system.  You might instead want the simulation to end
when you have completed a particular application workload (or collection of
application workloads), when a fault has been repaired, etc.  Talk about how
to handle this cleanly.

\subsection{Kicking off a simulation}

TOOD: fill this in.  Each LP needs to send an event to itself at the
beginning of the simulation (explain why).  We usually skew these with
random numbers to help break ties right off the bat (explain why).

\section{Best practices quick reference}

NOTE: these may be integrated with the remaining notes or used as a summary of
section(s).

\subsection{ROSS simulation development}

\begin{enumerate}

    \item prefer fine-grained, simple LPs to coarse-grained, complex LPs
    \begin{enumerate}
        \item can simplify both LP state and reverse computation implementation
        \item ROSS is very good at event processing, likely small difference in
            performance
    \end{enumerate}

    \item consider separating single-source generation of concurrent events with
        "feedback" events or "continue" events to self
    \begin{enumerate}
        \item generating multiple concurrent events makes rollback more difficult
    \end{enumerate}

    \item use dummy events to work around "event-less" advancement of simulation time 

    \item add a small amount of time "noise" to events to prevent ties

    \item prefer more and smaller events to fewer and larger events
    \begin{enumerate}
        \item simplifies individual event processing
        \item ROSS uses bounded event structure size in communication, so
            smaller bound $\rightarrow$  less communication overhead
    \end{enumerate}

    \item prefer placing state in event structure to LP state structure
    \begin{enumerate}
        \item simplifies reverse computation -- less persistent state
        \item NOTE: tradeoff with previous point - consider efficiency vs.
            complexity
    \end{enumerate}

    \item try to implement event processing with only LP-local information
    \begin{enumerate}
        \item reverse computation with collective knowledge is difficult
    \end{enumerate}

\end{enumerate}

\section{TODO}

\begin{itemize}
\item Build a single example model that demonstrates the concepts in this
document, refer to it throughout.
\item reference to ROSS user's guide, airport model, etc.
\item put a pdf or latex2html version of this document on the codes web page
when ready
\end{itemize}

\begin{figure}
\begin{lstlisting}[caption=Example code snippet., label=snippet-example]
for (i=0; i<n; i++) {
    for (j=0; j<i; j++) {
        /* do something */
    }
}
\end{lstlisting}
\end{figure}

Figure ~\ref{fig:snippet-example} shows an example of how to show a code
snippet in latex.  We can use this format as needed throughout the document.

\end{document}
