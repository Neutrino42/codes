
\documentclass[conference,10pt,compsocconf,onecolumn]{IEEEtran}
%\documentclass{acm_proc_article-sp}
%\documentclass{sig-alternate}
% Add the compsoc option for Computer Society conferences.
%
% If IEEEtran.cls has not been installed into the LaTeX system files,
% manually specify the path to it like:
% \documentclass[conference]{../sty/IEEEtran}

% *** CITATION PACKAGES ***
%
\usepackage{cite}
% cite.sty was written by Donald Arseneau
% V1.6 and later of IEEEtran pre-defines the format of the cite.sty package
% \cite{} output to follow that of IEEE. Loading the cite package will
% result in citation numbers being automatically sorted and properly
% "compressed/ranged". e.g., [1], [9], [2], [7], [5], [6] without using
% cite.sty will become [1], [2], [5]--[7], [9] using cite.sty. cite.sty's
% \cite will automatically add leading space, if needed. Use cite.sty's
% noadjust option (cite.sty V3.8 and later) if you want to turn this off.
% cite.sty is already installed on most LaTeX systems. Be sure and use
% version 4.0 (2003-05-27) and later if using hyperref.sty. cite.sty does
% not currently provide for hyperlinked citations.
% The latest version can be obtained at:
% http://www.ctan.org/tex-archive/macros/latex/contrib/cite/
% The documentation is contained in the cite.sty file itself.
%
\usepackage{setspace}
\usepackage{wrapfig}
\usepackage{color}

\usepackage[pdftex]{graphicx}
%\usepackage{graphicx}
% declare the path(s) where your graphic files are
\graphicspath{{./}{data/beagle/}}
% and their extensions so you won't have to specify these with
% every instance of \includegraphics
\DeclareGraphicsExtensions{.pdf,.jpeg,.png}

% *** GRAPHICS RELATED PACKAGES ***
%
\ifCLASSINFOpdf
 % \usepackage[pdftex]{graphicx}
 % declare the path(s) where your graphic files are
 % \graphicspath{{../pdf/}{../jpeg/}}
 % and their extensions so you won't have to specify these with
 % every instance of \includegraphics
 % \DeclareGraphicsExtensions{.pdf,.jpeg,.png}
\else
 % or other class option (dvipsone, dvipdf, if not using dvips). graphicx
 % will default to the driver specified in the system graphics.cfg if no
 % driver is specified.
 % \usepackage[dvips]{graphicx}
 % declare the path(s) where your graphic files are
 % \graphicspath{{../eps/}}
 % and their extensions so you won't have to specify these with
 % every instance of \includegraphics
 % \DeclareGraphicsExtensions{.eps}
\fi

% *** ALIGNMENT PACKAGES ***
%
%\usepackage{array}
% Frank Mittelbach's and David Carlisle's array.sty patches and improves
% the standard LaTeX2e array and tabular environments to provide better
% appearance and additional user controls. As the default LaTeX2e table
% generation code is lacking to the point of almost being broken with
% respect to the quality of the end results, all users are strongly
% advised to use an enhanced (at the very least that provided by array.sty)
% set of table tools. array.sty is already installed on most systems. The
% latest version and documentation can be obtained at:
% http://www.ctan.org/tex-archive/macros/latex/required/tools/

%\usepackage{eqparbox}
% Also of notable interest is Scott Pakin's eqparbox package for creating
% (automatically sized) equal width boxes - aka "natural width parboxes".
% Available at:
% http://www.ctan.org/tex-archive/macros/latex/contrib/eqparbox/

\usepackage{algorithm}
\usepackage[noend]{algorithmic}

% *** SUBFIGURE PACKAGES ***
\usepackage[tight,footnotesize]{subfigure}
% \usepackage{subfigure}
% subfigure.sty was written by Steven Douglas Cochran. This package makes it
% easy to put subfigures in your figures. e.g., "Figure 1a and 1b". For IEEE
% work, it is a good idea to load it with the tight package option to reduce
% the amount of white space around the subfigures. subfigure.sty is already
% installed on most LaTeX systems. The latest version and documentation can
% be obtained at:
% http://www.ctan.org/tex-archive/obsolete/macros/latex/contrib/subfigure/
% subfigure.sty has been superceeded by subfig.sty.

%\usepackage[caption=false]{caption}
%\usepackage[font=footnotesize]{subfig}
% subfig.sty, also written by Steven Douglas Cochran, is the modern
% replacement for subfigure.sty. However, subfig.sty requires and
% automatically loads Axel Sommerfeldt's caption.sty which will override
% IEEEtran.cls handling of captions and this will result in nonIEEE style
% figure/table captions. To prevent this problem, be sure and preload
% caption.sty with its "caption=false" package option. This is will preserve
% IEEEtran.cls handing of captions. Version 1.3 (2005/06/28) and later 
% (recommended due to many improvements over 1.2) of subfig.sty supports
% the caption=false option directly:
%\usepackage[caption=false,font=footnotesize]{subfig}
%
% The latest version and documentation can be obtained at:
% http://www.ctan.org/tex-archive/macros/latex/contrib/subfig/
% The latest version and documentation of caption.sty can be obtained at:
% http://www.ctan.org/tex-archive/macros/latex/contrib/caption/

% *** PDF, URL AND HYPERLINK PACKAGES ***
%
\usepackage{url}
% url.sty was written by Donald Arseneau. It provides better support for
% handling and breaking URLs. url.sty is already installed on most LaTeX
% systems. The latest version can be obtained at:
% http://www.ctan.org/tex-archive/macros/latex/contrib/misc/
% Read the url.sty source comments for usage information. Basically,
% \url{my_url_here}.

% *** Do not adjust lengths that control margins, column widths, etc. ***
% *** Do not use packages that alter fonts (such as pslatex).         ***
% There should be no need to do such things with IEEEtran.cls V1.6 and later.
% (Unless specifically asked to do so by the journal or conference you plan
% to submit to, of course. )

% correct bad hyphenation here
\hyphenation{op-tical net-works semi-conduc-tor}

\begin{document}
\title{CODES Best Practices}

%\author{\IEEEauthorblockN{Someone\IEEEauthorrefmark{1}} \\
%\IEEEauthorblockA{\IEEEauthorrefmark{1}Somewhere}
%}

%\numberofauthors{6} %  in this sample file, there are a *total*


% use for special paper notices
%\IEEEspecialpapernotice{(Invited Paper)}

% use arabic rather than roman numerals for table references
\renewcommand{\thetable}{\arabic{table}}

% make the title area
\maketitle

\begin{abstract}
This document outlines best practices for developing models in the
CODES/ROSS framework.  The reader should already be familiar with ROSS
and discrete event simulation; those topics are covered in the primary
ROSS documentation.
%
The main purpose of this document is to help the reader produce
CODES models in a consistent, modular style so that componets can be more
easily shared and reused.  It also includes a few tips to help avoid common
simulation bugs.
\end{abstract}

\section{CODES: modularizing models}

\subsection{Units of time}

use nanoseconds as units for time

\subsection{Organizing models by LP types}

split up distinct functionality (components of model) into different
LP types, give examples

\subsection{Protecting data structures}

don't expose event message or state structs across LP types.  Both
should be private types within the .c file that implements an LP.

\subsection{Techniques for notifying completion across LP types}

indicate completion across LP types by either delivering an opaque message
back to the calling LP, or by providing an API function for 2nd LP type to
use to call back (show examples of both)

\section{CODES: common utilities}

\subsection{codes\_mapping}

pull in Misbah's codes-mapping documentation

\subsection{modelnet}

\subsection{lp-io}

\section{CODES: reproducability and model safety}

\subsection{Event magic numbers}

\subsection{Small timestamps for LP transitions}

use codes\_local\_latency for timing of local event transitions

\section{ROSS: general tips}

\subsection{Organizing event structures}

using unions to clarify what fields in the event struct are used by each
event type in an LP

\subsection{Validating across simulation modes}

Check serial, conservative, and optimistic modes (all should work and give
consistent results)

\subsection{Reverse computation}

When to add it, some tips like keeping functions small, building
internal APIs with reverse functions, take advantage of ordering enforced by
ROSS, how to handle queues, etc.)

\section{TODO}

\begin{itemize}
\item reference to ROSS user's guide, airport model, etc.
\item figure out consistent way to format code snippets in document (just
reuse whatever we did in the Aesop paper)
\item put a pdf or latex2html version of this document on the codes web page
when ready
\end{itemize}

\end{document}
